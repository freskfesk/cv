%%%%%%%%%%%%%%%%%%%%%%%%%%%%%%%%%%%%%%%%%
% Medium Length Graduate Curriculum Vitae
% LaTeX Template
% Version 1.1 (9/12/12)
%
% This template has been downloaded from:
% http://www.LaTeXTemplates.com
%
% Original author:
% Rensselaer Polytechnic Institute (http://www.rpi.edu/dept/arc/training/latex/resumes/)
%
%%%%%%%%%%%%%%%%%%%%%%%%%%%%%%%%%%%%%%%%%

%----------------------------------------------------------------------------------------
%	PACKAGES AND OTHER DOCUMENT CONFIGURATIONS
%----------------------------------------------------------------------------------------

\documentclass[margin, 10pt, norsk]{res} % Use the res.cls style

\usepackage[utf8]{inputenc}
\usepackage[english, norsk]{babel}
\usepackage{setspace}

\usepackage[TS1,T1]{fontenc}
\usepackage{array, booktabs}
\usepackage{graphicx}
\usepackage[x11names]{xcolor}
\usepackage{colortbl}
\usepackage{caption}
\DeclareCaptionFont{blue}{\color{LightSteelBlue3}}

\usepackage{helvet} % Default font is the helvetica postscript font

\newcommand{\foo}{\color{LightSteelBlue3}\makebox[0pt]{\textbullet}\hskip-0.5pt\vrule width 1pt\hspace{\labelsep}}

\setlength{\textwidth}{5.1in} % Text width of the document

\begin{document}
	
	%----------------------------------------------------------------------------------------
	%	NAME AND ADDRESS SECTION
	%----------------------------------------------------------------------------------------
	
	\moveleft.5\hoffset\centerline{\LARGE\bf Vegard Grindheim} % Your name at the top
	
	\moveleft\hoffset\vbox{\hrule width\resumewidth height 1pt}\smallskip % Horizontal line after name
	
	
	\begin{table}[h]
		\begin{tabular}{lr}
			Nordeidåsen 6E  & 5251 Søreidgrend                \\
			vegard.grindheim@gmail.com & 991 66 981 
		\end{tabular}
	\end{table}
	
	%----------------------------------------------------------------------------------------
	
	\begin{resume}
		
		%----------------------------------------------------------------------------------------
		%	EDUCATION SECTION
		%----------------------------------------------------------------------------------------
		
		\hspace{10mm}  
		
		\section{UTDANNELSE}
		
		\textbf{\emph{Master i ingeniørfag, sivilingeniør} \hfill 2013 - 2015
			\\Teknisk kybernetikk}  \\
		Norges teknisk-naturvitenskapelige universitet \\
		Fordypning: Industrielle datamaskiner\\
		Masteroppgave: Precision Airdrop Using UAV 
		\begin{itemize} \itemsep -2pt % Reduce space between items
			\item Graden består av fordypning innen reguleringsteknikk, optimering, modelldesign, sanntidsprogrammering og programmering/design av innebygde systemer (embedded systems).
		\end{itemize}
		
		\textbf{\emph{Bachelor i ingeniørfag }\hfill 2010 - 2012\\
			Automatiseringsingeniør} \\
		Høgskolen i Bergen \\
		Bacheloroppgave: Robotisert tralle
		\begin{itemize} \itemsep -2pt % Reduce space between items
			\item Graden består av flere grunnfag innen fysikk, matematikk, elektronikk og programmering i tillegg til dypere kompetanse innen automatisering og regulering. 
		\end{itemize}
		
		\textbf{\emph{Master i ingeniørfag, sivilingeniør} \hfill 2009 - 2010
			\\Teknisk kybernetikk } \\
		Norges teknisk-naturvitenskapelige universitet
		\begin{itemize} \itemsep -2pt % Reduce space between items
			\item Fullførte første og andre semester.
		\end{itemize}
		
		\textbf{\emph{Studiespesialisering med realfag }\hfill 2006 - 2009\\}
		Danielsen videregående skole
		\begin{itemize} \itemsep -2pt % Reduce space between items
			\item Valgfag: Matematikk R1, Matematikk R2, Kjemi 1, Kjemi 2, Fysikk 1,\\Fysikk 2 og språkvalgfag Tysk II.
		\end{itemize}
		
		\newpage
		
		\hspace{5mm}  
		%----------------------------------------------------------------------------------------
		%	PROFESSIONAL EXPERIENCE SECTION
		%----------------------------------------------------------------------------------------
		
		\section{ERFARING}
		
		\textbf{\emph{Firmware Developer for Scanreach} \hfill 2023 - d.d.\\
			Firmwareutvikling}
		
		\begin{itemize} \itemsep -2pt % Reduce space between items
			\item Erfaring med integrert trådløst sensornettverk
			\item Scrum / smidig utvikling
			\item Programmering av mikrokontrollere fra Nordic Semiconductor og Texas Instruments
			\item Kodet i Visual Studio Code med støtte fra Copilot
			\item Kodespråk hovedsakelig C, men også Python og C\#
		\end{itemize}
		
		\textbf{\emph{Software Developer for Apparatus} \hfill 2021 - 2023\\
			Programvareutvikling}
		
		\begin{itemize} \itemsep -2pt % Reduce space between items
			\item Embedded programmering av STM mikrokontrollere
			\item Bruk av TouchGFX og CMSIS RTOS
			\item Kodet i C++
			\item Erfaring som eneste programvareutvikler
		\end{itemize}
		
		\textbf{\emph{Embedded Developer for Sensario} \hfill 2019 - 2021\\
			Programvareutvikling}
		
		\begin{itemize} \itemsep -2pt % Reduce space between items
			\item Embedded programmering av ARM mikrokontrollere
			\item Erfaring med Contiki OS
			\item Utvikling av sensorer til båtovervåkning
			\item Kodet hovedsakelig i C og C\#
		\end{itemize}
		
		\textbf{\emph{Senior Development Engineer for Data Respons} \hfill 2015 - 2019\\
			Programvareutvikling avdeling Bergen}
		
		\begin{itemize} \itemsep -2pt % Reduce space between items
			\item Smidig utvikling
			\item Utvikling av GUI, serverprogramvare, databaser
			\item Bred erfaring med Git, C++ og Qt
			\item Erfaring med Go, C\#, C, Ada, Assembly
		\end{itemize}
		
		\textbf{\emph{Disipliningeniør for REINERTSEN, deltid} \hfill 2013 - 2014\\
			Instrumenteringsingeniør, avdeling Trondheim}
		
		\begin{itemize} \itemsep -2pt % Reduce space between items
			\item Diverse arbeid med databaser og tegning
			\item Utledet arbeidspakker for offshore-personell 
		\end{itemize}
		
		\textbf{\emph{Disipliningeniør for REINERTSEN} \hfill 2012 - 2013 \\
			Instrumenteringsingeniør, avdeling Bergen\\
			Divisjon olje og gass.}
		\begin{itemize} \itemsep -2pt % Reduce space between items
			\item Utledet arbeidspakker for offshore-personell 
			\item Rekvirert instrumenter og annet utstyr
			\item Deltatt i store og små prosjekter på norsk sokkel
			\item Verdifull erfaring innen HMS
		\end{itemize}
		
		\newpage
		
		\hspace{5mm} 
		%----------------------------------------------------------------------------------------
		%	PROSJEKTER
		%----------------------------------------------------------------------------------------
		
		\section{PROSJEKTER}
		
		\textbf{\emph{Optimalisering av trådløst nettverk}}\\
		Det trådløse nettverket som brukes i Scanreach sine løsninger var utnyttet suboptimalt. For å støtte større nettverk gjorde jeg:
		\begin{itemize}\itemsep -2pt  % Reduce space between items 
			\item Måling og beregninger av kapasitet på nettverket
			\item Utviklet algoritmer for riktig utnyttelse av tilgjengelig båndbredde 
			\item Implementering og testing av algoritmene
		\end{itemize}
		
		\textbf{\emph{Smart IoT ølkokingsapparat for Apparatus}}\\
		Smart kjele for koking av øl. Jeg har vært med på utvikling av innebygd programvare (embedded).
		\begin{itemize}\itemsep -2pt  % Reduce space between items 
			\item PID-kontroller for styring av temperatur
			\item Portet AWS IoT-bibliotek
			\item Nedlasting av oppskrifter fra web-applikasjon
		\end{itemize}
		
		\textbf{\emph{Overvåkningssystem for båter for Sensario}}\\
		Overvåkningssystem som består av sensorer, embedded enhet, server, database og app. Jeg har vært med på utvikling av embedded enhet. Jeg har i hovedsak hatt ansvar for utvikling av:
		\begin{itemize}\itemsep -2pt  % Reduce space between items 
			\item Algoritme for vannivåmåling
			\item Automatisk geofence 
			\item Visning av akselerasjonsdata til bruker
			\item Lettvektskode for matematiske uttrykk
		\end{itemize}
		
		\textbf{\emph{Teknisk overvåkningssystem for Solberg Andersen}}\\
		Overvåkningssystem for ventiler. Systemet består av sensorer, embedded feltenheter, server, database og klient. Jeg har vært med på utvikling av server og klient fra 2016 til 2018. 
		\begin{itemize}\itemsep -2pt  % Reduce space between items
			\item  Visning av live data
			\subitem Data leses effektivt fra database og vises grafisk på skjerm i sanntid. 
			\subitem Dynamisk skalering av data
			\item Utvikling av effektive og presise algoritmer som finner nøkkeltall på måledata
			\item Konfigurasjon av maskinvare ved hjelp av dialoger
			\item Smidig utvikling hvor ny funksjonalitet ofte blir forespurt og håndtert fortløpende:
			\subitem Kravspesifikasjon
			\subitem Problemanalyse 
			\subitem Testing
			\subitem Utforming og implementasjon
		\end{itemize}
		
		\hspace{5mm} 
		%----------------------------------------------------------------------------------------
		%	REFERANSER
		%----------------------------------------------------------------------------------------
		
		\section{REFERANSER} 
		Referanser oppgis ved forespørsel.
		
	\end{resume}
\end{document}