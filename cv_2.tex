%%%%%%%%%%%%%%%%%%%%%%%%%%%%%%%%%%%%%%%%%
% Medium Length Graduate Curriculum Vitae
% LaTeX Template
% Version 1.1 (9/12/12)
%
% This template has been downloaded from:
% http://www.LaTeXTemplates.com
%
% Original author:
% Rensselaer Polytechnic Institute (http://www.rpi.edu/dept/arc/training/latex/resumes/)
%
% Important note:
% This template requires the res.cls file to be in the same directory as the
% .tex file. The res.cls file provides the resume style used for structuring the
% document.
%
%%%%%%%%%%%%%%%%%%%%%%%%%%%%%%%%%%%%%%%%%

%----------------------------------------------------------------------------------------
%	PACKAGES AND OTHER DOCUMENT CONFIGURATIONS
%----------------------------------------------------------------------------------------

\documentclass[margin, 10pt, norsk]{res} % Use the res.cls style, the font size can be changed to 11pt or 12pt here

\usepackage[utf8]{inputenc}
\usepackage[english, norsk]{babel}
\usepackage{setspace}

\usepackage[TS1,T1]{fontenc}
\usepackage{array, booktabs}
\usepackage{graphicx}
\usepackage[x11names]{xcolor}
\usepackage{colortbl}
\usepackage{caption}
\DeclareCaptionFont{blue}{\color{LightSteelBlue3}}

\usepackage{helvet} % Default font is the helvetica postscript font

\newcommand{\foo}{\color{LightSteelBlue3}\makebox[0pt]{\textbullet}\hskip-0.5pt\vrule width 1pt\hspace{\labelsep}}

\setlength{\textwidth}{5.1in} % Text width of the document

\begin{document}

%----------------------------------------------------------------------------------------
%	NAME AND ADDRESS SECTION
%----------------------------------------------------------------------------------------

\moveleft.5\hoffset\centerline{\LARGE\bf Vegard Grindheim} % Your name at the top
 
\moveleft\hoffset\vbox{\hrule width\resumewidth height 1pt}\smallskip % Horizontal line after name; adjust line thickness by changing the '1pt'
 

\begin{table}[h]
\begin{tabular}{lr}
Nordeidåsen 6E  & 5251 Søreidgrend               \\
vegard.grindheim@hotmail.com & 991 66 981 
\end{tabular}
\end{table}
 
%----------------------------------------------------------------------------------------

\begin{resume}

%----------------------------------------------------------------------------------------
%	OBJECTIVE SECTION
%----------------------------------------------------------------------------------------
 
%\section{OBJECTIVE}  

%A position in the field of computers with special interests in business applications programming, information processing, and %management systems. 

%----------------------------------------------------------------------------------------
%	EDUCATION SECTION
%----------------------------------------------------------------------------------------

\section{UTDANNELSE}

\textbf{\emph{Master i ingeniørfag, sivilingeniør} \hfill 2013 - 2015
\\Teknisk kybernetikk  \\
Norges teknisk-naturvitenskaplige universitet \\
Fordypning: Industrielle datamaskiner\\
Masteroppgave: Precision Airdrop Using UAV   } 
\begin{itemize} \itemsep -2pt % Reduce space between items
\item Graden består av fordypning innen reguleringsteknikk, optimering, modelldesign, sanntidsprogrammering og programmering/design av innbakte systemer (embedded systems).
\end{itemize}

\textbf{\emph{Bachelor i ingeniørfag }\hfill 2010 - 2012\\
Automatiseringsingeniør \\
Høgskolen i Bergen \\
Bacheloroppgave: Robotisert tralle  } %FIXFIXFIXFIXFIXFIXFIXFIXFIX
\begin{itemize} \itemsep -2pt % Reduce space between items
\item Graden består av flere grunnfag innen fysikk, matematikk, elektronikk og programmering i tillegg til dypere kompetanse innen automatisering og regulering. 
\end{itemize}

\textbf{\emph{Master i ingeniørfag, sivilingeniør} \hfill 2009 - 2010
\\Teknisk kybernetikk  \\
Norges teknisk-naturvitenskaplige universitet} 
\begin{itemize} \itemsep -2pt % Reduce space between items
\item Første og andre semester,
\end{itemize}

\textbf{\emph{Studiespesialisering med realfag }\hfill 2006 - 2009\\
Danielsen videregående skole}
\begin{itemize} \itemsep -2pt % Reduce space between items
\item Valgfag: Matematikk R1, Matematikk R2, Kjemi 1, Kjemi 2, Fysikk 1,\\Fysikk 2 og språkvalgfag Tysk II.
\end{itemize}

\hspace{5mm}  
%----------------------------------------------------------------------------------------
%	PROFESSIONAL EXPERIENCE SECTION
%----------------------------------------------------------------------------------------

\section{ERFARING}

\textbf{\emph{Development Engineer for Data Respons} \hfill 2015 - D.d.\\
Programvareutvikling avdeling Bergen}

\begin{itemize} \itemsep -2pt % Reduce space between items
\item Smidig utvikling
\item Utvikling av GUI, serverprogramvare, databaser
\item Grundig erfaring med Git, C++ og Qt
\item Erfaring med Go, C\#, C, Ada, Assembly
\end{itemize}

\textbf{\emph{Disipliningeniør for REINERTSEN, deltid} \hfill 2013 - 2014\\
Instrumentingeniør avdeling Trondheim
Divisjon olje og gass.}

\begin{itemize} \itemsep -2pt % Reduce space between items
\item Diverse arbeid med databaser og tegning
\item Utledet arbeidspakker for offshore personell 
\end{itemize}

\newpage

\textbf{\emph{Disipliningeniør for REINERTSEN} \hfill 2012 - 2013 \\
Instrumentingeniør avdeling Bergen\\
Divisjon olje og gass.}
\begin{itemize} \itemsep -2pt % Reduce space between items
\item Utledet arbeidspakker for offshore personell 
\item Rekvirert instrumenter og annet utstyr
\item Deltatt i store og små prosjekter på norsk sokkel
\item Verdifull erfaring innen HMS
\end{itemize}

\hspace{5mm} 
%----------------------------------------------------------------------------------------
%	Prosjekter
%----------------------------------------------------------------------------------------

\section{PROSJEKTER}

\textbf{\emph{Utvikling av teknisk overvåkningsprogram for oljebransjen}}\\
Overvåkningsprogram som består av sensorer, embedded feltenheter, server, database og klient. Jeg har vært med på utvikling av server og klient siden mai 2016. 
\begin{itemize}\itemsep -2pt   % Reduce space between items
\item  Visning av live data
\subitem Live data blir effektivt lest fra database og vist grafisk på skjerm i realtime. 
\subitem Dynamisk skalering av data
\item Utvikle effektive og presise algoritmer som finner spesifikke punkter på en graf
\item Konfigurasjon av hardware ved hjelp av dialoger
\item Smidig utvikling hvor ny funksjonalitet ofte blir forespurt, da blir det utført kontinuerlig
\subitem Kravspesifikasjon
\subitem Problemanalyse 
\subitem Testing
\subitem Utforming og implementasjon
\end{itemize}

\textbf{\emph{Utvikling av CLI for testutstyr}}\\
Konsollbasert program for å forenkle testing av utviklet hardware. Engasjementet som konsulent varte i 2 måneder. 
\begin{itemize}\itemsep -2pt   % Reduce space between items
\item  Skrevet i C
\item Bruk av SPI-buss for å kommunisere med elektronikk.
\end{itemize}

\textbf {\emph{Videreutvikling av intern programvare}}\\
Bugfiksing av program brukt til å holde orden på materiell. Løpende utvikling, 6 måneder.
\begin{itemize}\itemsep -2pt   % Reduce space between items
\item   Skrevet i C\# med Visual Studio
\end{itemize}

\textbf {\emph{Klient og server}}\\
Skoleprosjekt i .Net som talte 5 studiepoeng
\begin{itemize}\itemsep -2pt   % Reduce space between items
\item   Skrevet i C\# med Visual Studio
\item Omfattet servere og klienter med dynamisk tilkobling på et lokalt nett
\end{itemize}

\hspace{5mm} 
%----------------------------------------------------------------------------------------
%	Referanser
%----------------------------------------------------------------------------------------

\section{REFERANSER} 
Referanser kan skaffes ved behov

%\textbf{Gisle Skarphagen\hfill Mob.: 907 27 793 }\\ Disiplinleder, instrument, REINERTSEN Bergen 
%\textbf{Trond Håkon Sandvoll\hfill Mob.: 930 92 320}\\ Disiplinleder, anlegg, REINERTSEN Bergen 
%\textbf{Thor Inge Fossen\hfill Mob.: 918 97 361}\\ Professor i navigasjon og fartøystyring, NTNU

%----------------------------------------------------------------------------------------


\end{resume}
\end{document}